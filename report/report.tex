\documentclass{article}
\usepackage{graphicx}
\usepackage{fullpage}
\usepackage{amsmath}
\usepackage{hyperref}


\title{Parallel domain decomposition in 2D}
\author{Gavin Stewart and Sanchit Chaturvedi}

\begin{document}
	\maketitle
	
	\section{Introduction}
	 
	Elliptic problems are important in applied and mathematical physics.  The Poisson equation 
	\begin{equation}
		-\Delta u = f
	\end{equation} describes the voltage \(u\) due to a charge distribution with density \(f\)%REF
	, as well as describing the pressure in incompressible fluid flow \cite{Marshall97}. The Helmholtz equation
	\begin{equation}
		(-\Delta u + k^2)u = f
	\end{equation} 
	describes the propagation of electromagnetic or acoustic waves \cite{Fairweather03}.
	
	For large scale problems, it is desirable to parallelize the computation of solutions.  This is difficult for elliptic problems due to the strong coupling between unknowns when \(u\) is computed through finite difference or finite element methods.  Domain decomposition methods split the computational domain into several regions (which may or may not overlap), compute an estimate of the solution on each  subdomain, and then combine the solutions on each subdomain to get an estimate of the new solution.  The choice of boundary conditions for the subdomains varies between different methods%REF
	; we will use a method with Dirichlet boundary conditions.
	
	The layout of the report is as follows: in section 2, we will describe the domain decomposition algorithm in detail.  In section 3, we will give weak and strong scaling results for an implementation of the algorithm with \(0\) overlap.  In section 4, we give an explanation for the slow convergence of the method, and suggest possible improvements to speed up the method.  Finally, section 5 gives an account of both authors' contributions to the project.
	
	\section{The algorithm}
	
	\section{Scaling results}
	\textbf{Weak Scaling}: The local grid is fixed to be $100\times100$ and the number of processors are assumed to be a perfect square with $max iter=500$. 
	
	\includegraphics[scale=.4]{/Users/sanchit/Desktop/high_performace_computing/repository/finalproj/weak_scaling.jpg}
	
	As is evident from the plot the problem scales well. The increase in time is mostly because of increased communication because we are using a local solver(UMFPACK) to solve the poisson's equation locally which works in the same time as the local problem size is fixed.
	
	\textbf{Strong Scaling}: The global grid is fixed at $3600\times3600$ and the number of processors are assumed to be a perfect square with $max iter=100$.
	
		\includegraphics[scale=.4]{/Users/sanchit/Desktop/high_performace_computing/repository/finalproj/strong_scaling.jpg}

The plot suggests that the problem scales strongly almost perfectly. 
	\section{Possible improvements}
	\begin{itemize}
\item	The strong scaling demonstrates that the convergence is very slow, both in time and in residual as well. This is in agreement with the theory that suggesgs that the converence is proportional to the overlap between the domains. So a possible improvement would be to implement the code that can tackle the overlapping domains. 
\item Another possible direction to improve would be to coupling the multigrid or multilevel methods to get a better convergence rate.
\item Nonoverlapping method can be used as a preconditioner for GMRES or CG, so exploring this direction is another option for future studies.
	\end{itemize}

	\section{Author contributions}
	
	\begin{thebibliography}{9}
		\bibitem{Marshall97}
		Marshall, J., Adcroft, A., Hill, C., Perelman, L., \& Heisey, C.
		(1997).
		\emph{A finite-volume, incompressible Navier Stokes model for studies of the ocean on parallel computers}.
		Journal of Geophysical Research 
		102(C3), 5753--5766.
		
		\bibitem{Fairweather03}
		Fairweather, G., Karageorghis, A., \& Martin, P.A.
		(2003).
		\emph{The method of fundamental solutions for scattering and radiation problems}.
		Engineering analysis with boundary elements   
		27(7), 759--769.
		\url{https://doi.org/10.1016/S0955-7997(03)00017-1}
		
	\end{thebibliography}
\end{document}